\documentclass[aspectratio=169]{beamer}
\usepackage{listings}
\usepackage{fancyvrb}
\usepackage{xcolor}

\usepackage{tikz}
\usetikzlibrary{shapes,arrows,positioning}
\usepackage{amsmath,bm,times}
\newcommand{\mx}[1]{\mathbf{\bm{#1}}} % Matrix command
\newcommand{\vc}[1]{\mathbf{\bm{#1}}} % Vector command

\lstdefinestyle{base}{
    language=C,
    emptylines=1,
    breaklines=true,
    basicstyle=\ttfamily\color{black},
    moredelim=**[is][\color{red}]{@}{@},
}

\newcommand{\tabitem}{%
  \usebeamertemplate{itemize item}\hspace*{\labelsep}}

\usetheme{metropolis}           % Use metropolis theme
\title{The nix package manager}
\subtitle{Declarative package management on Linux/OSX}
\date{March 8, 2018}
\author{Tobias Mayer}
\institute{TomTom Development Germany}
\begin{document}
\maketitle

\begin{frame}[label=package]{Motivation}
    \begin{columns}[t]
        \begin{column}{0.4\textwidth}
            \begin{overlayarea}{\textwidth}{.4\textheight}
                language (pip, cargo)
                \smallskip
                \only<2->{
                \begin{itemize}
                    \item[+] {declarative}
                    \item[-] {build time / run time discrepancy}
                    \item[-] {\color{red}incomplete}
                \end{itemize}
                }
            \end{overlayarea}
        \end{column}

        \begin{column}{0.4\textwidth}
            \begin{overlayarea}{\textwidth}{.4\textheight}
                system (apt, dnf, brew)
                \smallskip
                \only<3->{
                \begin{itemize}
                    \item[+] {self-sufficient}
                    \item[-] {tied to platform}
                    \item[-] {\color{red}stateful}
                    %\item[-] {\color{red}package creation}
                \end{itemize}
                }
            \end{overlayarea}
        \end{column}

    \end{columns}
    \begin{overlayarea}{\textwidth}{.2\textheight}
    \only<4->{
        \begin{center}
            \begin{minipage}{0.4\textwidth}
                \begin{itemize}
                    \item[-] {\color{red}not reproducible}
                    \item[-] {\color{red}not deterministic}
                \end{itemize}
            \end{minipage}
        \end{center}
    }
    \end{overlayarea}
\end{frame}

%\begin{frame}{Motivation 1}
%    System level package management (apt, dnf): \\
%    \begin{itemize}
%        \item stateful
%        \item heterogeneous set of packages
%        \item packaging often underspecified
%    \end{itemize}
%\end{frame}
%
%\begin{frame}{Motivation 2}
%    Language level package management (pip, npm, cargo): \\
%    \begin{itemize}
%        \item does not provide platform
%        \item does not provide non-native dependencies
%        \item build time / run time discrepancy
%    \end{itemize}
%\end{frame}

\begin{frame}<1,2>[label=ecosystem]{The nix ecosystem}
    \begin{columns}[t]
        \begin{column}{0.4\textwidth}
            \begin{overlayarea}{\textwidth}{.5\textheight}
                \begin{itemize}
                        \only<1>{\item package manager}
                        \only<2->{\item \textbf{package manager}}
                        \only<1>{\item expression language}
                        \only<2->{\item \textbf{expression language}}
                        \only<1>{\item nixpkgs}
                        \only<2->{\item \textbf{nixpkgs}}
                        \only<1->{\item nixos}
                        \only<1->{\item nixops}
                        \only<1->{\item disnix}
                        \only<1->{\item hydra}
                \end{itemize}
            \end{overlayarea}
        \end{column}
        \begin{column}{0.4\textwidth}
            \begin{overlayarea}{\textwidth}{.5\textheight}
                \includegraphics[width=\linewidth]{nixos-logo-only-hires.png}
            \end{overlayarea}
        \end{column}
    \end{columns}
\end{frame}

\begin{frame}{Approach}
    \small
    \begin{itemize}
            \only<1->{\item Source based distribution
            \item Unique per-package install location \\
                /nix/store/\textit{hash-name-version \\
            example: /nix/store/86blj9iqyxwmdgkn3dyrpib1gkbmz91v-bash-4.4-p5}}
    \only<2->{\item[->] multiple variants of a package available simultaneously}
    \only<3->{\item[->] no duplication}
    \only<4->{\item[->] transparent binary caching}
    \end{itemize}


\end{frame}

\begin{frame}[label=cmdl]{The nix package manager}
    \begin{overlayarea}{\textwidth}{.7\textheight}
        Commands
        \textbf{
        \begin{itemize}
                {\item nix-instantiate}
                {\item nix-store}
                {\item nix-shell}
        \end{itemize}
        }
        \begin{itemize}
                {\item nix-build}
                {\item nix-env}
                {\item nix-collect-garbage}
                {\item nix-hash}
                {\item …}
                {\item nix (added with version 2, currently beta)}
        \end{itemize}
    \end{overlayarea}
\end{frame}

\begin{frame}[label=nixpkgs]{nixpkgs}
    \begin{itemize}
        \item github.com/NixOS/nixpkgs
        \item Public collection of nix expressions
        \item sorted into channels:
            \begin{itemize}
                \item nixpkgs-unstable
                \item nixos-unstable
                \item nixos-17.09
                \item nixos-18.03
                \item …
            \end{itemize}
    \end{itemize}
\end{frame}

\begin{frame}[fragile,label=eexpr]{example expression}
    \small
\begin{lstlisting}[style=base,breaklines=false]
1  { @stdenv@, @fetchurl@, @perl@ }:
2
3  @stdenv@.mkDerivation {
4    name = "hello-2.10";
5    builder = ./builder.sh;
6    src = @fetchurl@ {
7      url = ftp://ftp.nl…/
8                hello-2.10.tar.gz;
9      sha256 = "1md7jsfd8pa…";
10   };
11   buildInputs = [ @perl@ ];
11 }
\end{lstlisting}
\end{frame}

\begin{frame}[fragile]{.drv file}
    \footnotesize
\begin{lstlisting}[style=base,breaklines=false]
Derive (
  [("out", "/nix/store/kmwd1~hello-2.10", "", "")],
  [("/nix/store/2zy7f~hello-2.10.tar.gz.drv", ["out"]),
   ("/nix/store/dxlr6~stdenv.drv", ["out"]),
   ("/nix/store/qkf7y~bash-4.4-p12.drv", ["out"]),
   ("/nix/store/wy1na~perl-5.24.3.drv", ["out"])],
  ["/nix/store/9krlz~default-builder.sh"],
  "x86_64-linux",
  "/nix/store/zqh3l~bash-4.4-p12/bin/bash",
  ["-e", "/nix/store/9krlz~default-builder.sh"],
  [("buildInputs", ""),
   ("builder", "/nix/store/zqh3l~bash-4.4-p12/bin/bash"),
   ("stdenv", "/nix/store/l3nz0~stdenv"),
   ...
   ("system", "x86_64-linux")]
)
\end{lstlisting}
\end{frame}

\begin{frame}[fragile]{.drv file (outputs)}
    \footnotesize
\begin{lstlisting}[style=base,breaklines=false]
Derive (
  @[("out", "/nix/store/kmwd1~hello-2.10", "", "")]@,
  [("/nix/store/2zy7f~hello-2.10.tar.gz.drv", ["out"]),
   ("/nix/store/dxlr6~stdenv.drv", ["out"]),
   ("/nix/store/qkf7y~bash-4.4-p12.drv", ["out"]),
   ("/nix/store/wy1na~perl-5.24.3.drv", ["out"])],
  ["/nix/store/9krlz~default-builder.sh"],
  "x86_64-linux",
  "/nix/store/zqh3l~bash-4.4-p12/bin/bash",
  ["-e", "/nix/store/9krlz~default-builder.sh"],
  [("buildInputs", ""),
   ("builder", "/nix/store/zqh3l~bash-4.4-p12/bin/bash"),
   ("stdenv", "/nix/store/l3nz0~stdenv"),
   ...
   ("system", "x86_64-linux")]
)
\end{lstlisting}
\end{frame}

\begin{frame}[fragile]{.drv file (inputs)}
    \footnotesize
\begin{lstlisting}[style=base,breaklines=false]
Derive (
  [("out", "/nix/store/kmwd1~hello-2.10", "", "")],
  @[("/nix/store/2zy7f~hello-2.10.tar.gz.drv", ["out"]),
   ("/nix/store/dxlr6~stdenv.drv", ["out"]),
   ("/nix/store/qkf7y~bash-4.4-p12.drv", ["out"]),
   ("/nix/store/wy1na~perl-5.24.3.drv", ["out"])],
  ["/nix/store/9krlz~default-builder.sh"]@,
  "x86_64-linux",
  "/nix/store/zqh3l~bash-4.4-p12/bin/bash",
  ["-e", "/nix/store/9krlz~default-builder.sh"],
  [("buildInputs", ""),
   ("builder", "/nix/store/zqh3l~bash-4.4-p12/bin/bash"),
   ("stdenv", "/nix/store/l3nz0~stdenv"),
   ...
   ("system", "x86_64-linux")]
)
\end{lstlisting}
\end{frame}

\begin{frame}[fragile]{.drv file (builder+arguments)}
    \footnotesize
\begin{lstlisting}[style=base,breaklines=false]
Derive (
  [("out", "/nix/store/kmwd1~hello-2.10", "", "")],
  [("/nix/store/2zy7f~hello-2.10.tar.gz.drv", ["out"]),
   ("/nix/store/dxlr6~stdenv.drv", ["out"]),
   ("/nix/store/qkf7y~bash-4.4-p12.drv", ["out"]),
   ("/nix/store/wy1na~perl-5.24.3.drv", ["out"])],
  ["/nix/store/9krlz~default-builder.sh"],
  "x86_64-linux",
  @"/nix/store/zqh3l~bash-4.4-p12/bin/bash",
  ["-e", "/nix/store/9krlz~default-builder.sh"]@,
  [("buildInputs", ""),
   ("builder", "/nix/store/zqh3l~bash-4.4-p12/bin/bash"),
   ("stdenv", "/nix/store/l3nz0~stdenv"),
   ...
   ("system", "x86_64-linux")]
)
\end{lstlisting}
\end{frame}

\begin{frame}[fragile]{.drv file (environment variables)}
    \footnotesize
\begin{lstlisting}[style=base,breaklines=false]
Derive (
  [("out", "/nix/store/kmwd1~hello-2.10", "", "")],
  [("/nix/store/2zy7f~hello-2.10.tar.gz.drv", ["out"]),
   ("/nix/store/dxlr6~stdenv.drv", ["out"]),
   ("/nix/store/qkf7y~bash-4.4-p12.drv", ["out"]),
   ("/nix/store/wy1na~perl-5.24.3.drv", ["out"])],
  ["/nix/store/9krlz~default-builder.sh"],
  "x86_64-linux",
  "/nix/store/zqh3l~bash-4.4-p12/bin/bash",
  ["-e", "/nix/store/9krlz~default-builder.sh"],
  @[("buildInputs", ""),
   ("builder", "/nix/store/zqh3l~bash-4.4-p12/bin/bash"),
   ("stdenv", "/nix/store/l3nz0~stdenv"),
   ...
   ("system", "x86_64-linux")]@
)
\end{lstlisting}
\end{frame}

\begin{frame}[label=store]{The store (conceptual)}
    \only<1>{initial state}
    \only<2>{nix-instantiate -E 'with import <nixpkgs> \{ \}; hello'}
    \only<3>{nix-store -r /nix/store/g85g3\char`~hello-2.10.drv}
    \only<4>{ln -s /nix/store/g85g3\char`~hello-2.10.drv /nix/var/nix/gcroots/}
    \only<5>{nix-store -gc}
    \only<6>{rm /nix/var/nix/gcroots/g85g3\char`~hello-2.10.drv}
    \only<7>{nix-store -gc}
    \only<8>{final state}
    \center
    \only<1-8>{
    \begin{tikzpicture}
        % We need layers to draw the block diagram
        \pgfdeclarelayer{background}
        \pgfdeclarelayer{foreground}
        \pgfsetlayers{background,main,foreground}

        % Define a few styles and constants
        \tikzstyle{pilp} = [->, thick, color=blue!50, text width=5em]
        \def\blockdist{2.3}
        \def\edgedist{2.5}

        \tikzstyle{drvp}=[draw, fill=blue!20, text width=7em,
             text centered, minimum height=1em, font=\tiny]
        \tikzstyle{bop}=[draw, fill=red!20, text width=7em,
             text centered, minimum height=1em, font=\tiny]
        \only<1,8->{\tikzstyle{drv}=[text width=7em,
                 text centered, minimum height=1em, font=\tiny]}
        \only<2-6>{\tikzstyle{drv}=[fill=blue!20, text width=7em,
                 text centered, minimum height=1em, font=\tiny]}
        \only<7>{\tikzstyle{drv}=[fill=gray!20, text width=7em,
                 text centered, minimum height=1em, font=\tiny]}
        \only<1-2,8->{\tikzstyle{bo}=[text width=7em,
                 text centered, minimum height=1em, font=\tiny]}
        \only<3-6>{\tikzstyle{bo}=[fill=red!20, text width=7em,
                 text centered, minimum height=1em, font=\tiny]}
        \only<7>{\tikzstyle{bo}=[fill=gray!20, text width=7em,
                 text centered, minimum height=1em, font=\tiny]}

        \only<1-6>{\tikzstyle{pil} = [->, thick, color=blue!50, text width=5em]}
        \only<7->{\tikzstyle{pil} = [->, thick, color=gray!50, text width=5em]}

        \node (gcc7) [drvp] {jb6h0\char`~gcc-7.3.0.drv};
        \node (nethack) [drvp, below=2mm of gcc7] {8v9py\char`~nethack-3.6.0.drv};
        \only<1,8>{
            \node (gnumake) [drv, below=2mm of nethack] {};
            \node (perl) [drv, below=2mm of gnumake] {};
            \node (hello) [drv, below=2mm of perl] {};
        }
        \only<2-7>{
            \node (gnumake) [draw,drv, below=2mm of nethack] {b1imi\char`~gnumake-4.2.1.drv};
        \node (perl) [draw,drv, below=2mm of gnumake] {wy1na\char`~perl-5.24.3.drv};
        \node (hello) [draw,drv, below=2mm of perl] {g85g3\char`~hello-2.10.drv};
        }

        \only<2-7>{
        \path[every node/.style={font=\sffamily\small,
              fill=white,inner sep=1pt}]
           (gnumake.west)
             edge [pil, bend left=40] (gcc7.west)

           (perl.west)
             edge [pil, bend left=40] (gnumake.west)
             edge [pil, bend left=40] (gcc7.west)

           (hello.west)
             edge [pil, bend left=40] (perl.west)
             edge [pil, bend left=40] (gnumake.west)
             edge [pil, bend left=40] (gcc7.west);
         }

        \path[every node/.style={font=\sffamily\small,
              fill=white,inner sep=1pt}]
          (nethack.west)
             edge [pilp, bend left=40] (gcc7.west);

        \node (bgcc7) [bop, right=of gcc7] {7ygz3\char`~gcc-7.3.0};
        \node (bnethack) [bop, right=of nethack] {v237i\char`~nethack-3.6.0};
        \only<1,2,8>{
            \node (bgnumake) [bo, right=of gnumake] {};
            \node (bperl) [bo, right=of perl] {};
            \node (bhello) [bo, right=of hello] {};
        }
        \only<3-7>{
            \node (bgnumake) [draw, bo, right=of gnumake] {lhp5r\char`~gnumake-4.2.1};
            \node (bperl) [draw, bo, right=of perl] {93865\char`~perl-5.24.3};
            \node (bhello) [draw, bo, right=of hello] {kmwd1\char`~hello-2.10};
        }

        \path [draw, pilp] (gcc7) -- (bgcc7);
        \path [draw, pilp] (nethack) -- (bnethack);
        \only<3-7>{
        \path [draw, pil] (gnumake) -- (bgnumake);
        \path [draw, pil] (perl) -- (bperl);
        \path [draw, pil] (hello) -- (bhello);
    }

    \node (drvs) [below of=hello] {derivations};
        \node (bos) [below of=bhello] {outputs};

        \path (gcc7) -- (bgcc7) node[midway] (auxlogical) {};
        \node (logical) [above=0.5 of auxlogical] {\textbf{/nix/store}};

        \path (hello.west |- drvs.south) node (ankgarbage) {};
        \node (garbage) [below=7mm of ankgarbage, anchor=west] {\textbf{/nix/var/nix/gcroots/…}};
        \path (garbage.east)+(0.5,0) node (gcl1) {};
        \path (gcl1)+(0.5,0) node (gcl2) {};

        \path[every node/.style={font=\sffamily\small,
              fill=white,inner sep=1pt}]
           (gcl2)
           edge [pilp, dashed, bend right=10] (nethack.south);

        \only<4-5>{\path[every node/.style={font=\sffamily\small,
              fill=white,inner sep=1pt}]
           (gcl1)
           edge [pil, dashed, bend right=10] (hello.south);}

        % Now it's time to draw the colored IMU and INS rectangles.
        % To draw them behind the blocks we use pgf layers. This way we
        % can use the above block coordinates to place the backgrounds
        \begin{pgfonlayer}{background}
            % Compute a few helper coordinates
            \path (gcc7.west |- logical.north)+(-0.9,0.3) node (a) {};
            \path (bos.south -| bhello.east)+(+0.5,-0.2) node (b) {};
            \path[fill=yellow!20,rounded corners, draw=black!50, dashed]
                (a) rectangle (b);
            \path (gcc7.north west)+(-0.7,0.2) node (a) {};
            \path (drvs.south -| gcc7.east)+(+0.2,-0.1) node (b) {};
            \path[fill=blue!10,rounded corners, draw=black!50, dashed]
                (a) rectangle (b);
            \path (bgcc7.north west)+(-0.2,0.2) node (a) {};
            \path (bos.south -| bgcc7.east)+(+0.2,-0.1) node (b) {};
            \path[fill=red!10,rounded corners, draw=black!50, dashed]
                (a) rectangle (b);
            \path (gcc7.west |- garbage.north)+(-0.7,0.2) node (a) {};
            \path (garbage.south -| bhello.east)+(0.2,-0.2) node (b) {};
            \path[rounded corners, draw=black!50, dashed]
                (a) rectangle (b);
        \end{pgfonlayer}
    \end{tikzpicture}
    }
\end{frame}

\begin{frame}{expression language}
    \begin{itemize}
        \item functional
        \item pure
        \item lazy
    \end{itemize}
    \begin{itemize}
        \item numbers, booleans, strings, paths
        \item lists: [3 "hello" true]
        \item sets: \{ x = 123; text = "Hi!"; y = f \{ foo = "bar"; \}; \}
        \item function def:       funcname = pattern: body
        \item function call:      funcname argument
        \item function def (2):   funcname = \{ pattern1, pattern2 ? defaultval \}: body
        \item function call (2):  funcname \{ argument set \}
    \end{itemize}
\end{frame}

%\againframe{eexpr}

\begin{frame}[fragile]{nix-shell expression}
    shell.nix:
    \small
\begin{lstlisting}[style=base,breaklines=false]
{}:
let
  pkgs = import (fetchTarball
    https://github.com/NixOS/nixpkgs/archive/17.03.tar.gz) {};
in with pkgs; {
  environment = stdenv.mkDerivation {
    name = "myProject";
    nativeBuildInputs = [ cmake ];
    buildInputs = [ boost ];
  };
}
\end{lstlisting}
use with 'nix-shell --pure --indirect --add-root gcroots/shell.gc'
\end{frame}

\begin{frame}{advanced topics}
    \begin{itemize}
        \item parameterized shells
        \item link time deps with the same toolset
        \item write expressions for dependencies
    \end{itemize}
\end{frame}

\begin{frame}[c]{fin}
\centering
Thank you! \\
github.com/tobimpub/presentations/20180308-devday-nix
\end{frame}
\end{document}
